\documentclass{article}
\usepackage{graphicx}
\usepackage[margin=0.5in]{geometry}
\PassOptionsToPackage{hyphens}{url}\usepackage{hyperref}
\begin{document}

\title{Command and Control Subsystems Report}
\author{Jake Vossen: OREPACKAGERS}

\maketitle

% \begin{abstract}
% The abstract text goes here.
% \end{abstract}

\section{Subsystem Description}
The Command and Control subsystem is the subsystem responsible for
converting the requests that have been collected into downloaded data
to be distributed to users. It starts by receiving a list of
\texttt{request} objects - a structure for contating information about
each request. To prevent confusion, the mono-spaced \texttt{request}
will refer to they Python object itself, wheras plain ``request''
refers to the concept of a user request.

With this list of requests, the first thing it does is use Pythons
\texttt{multiprocessing}[1] library to split work up between the
different threads on the computer. While this software is designed
for low end machines to be more accessable to developing areas, most
computers[2] in recent times will have more than 1 CPU core (including
the Raspberry Pi[3]). This allows for the processor to split up all
the requests, and execute them in paralel, instead of waiting for each
one to finish individually, which can provide a large preformance
boost.

When downloading a \texttt{request}, it determens the type of
request. The types are URL, search, youtube, and ipfs. The steps for
each type of request is outlined below. 

\subsection{URLs}

URLs are your basic websites, such as
\url{https://en.wikipedia.org/wiki/Monty_Python_and_the_Holy_Grail},
or
\url{https://www.nytimes.com/2019/03/27/technology/turing-award-ai.html}. This
is for users who already know the content they want. In the backend,
the Python program is going to use the \texttt{wget}[4]
utility. Specifically, \texttt{wget -E -H -k -K -p -P path url
  robots=off} where \texttt{path} is the output directory and
\texttt{url} is the url that has been requested. To break it down:
\begin{itemize}
  \item \texttt{-E} tells \texttt{wget} to change the file extention
    if the url isn't a .html file. This allows for the downloading of
    PDF files as well as HTML files
  \item \texttt{-H} Tells \texttt{wget} that it is okay to download
    material from hosts that aren't from the specified URL. While this
    seems backwards at first, many websites host their fonts or
    pictures in a place that isn't the same as the document that is
    being request. This allows the page to appear just as it would
    when visited in a web browser
  \item \texttt{-k} This stands for ``convert links'', which means
    that when the download is complete, it converts the links on the
    page so they are sudible for browsing on the local machine. For
    example, if a blog has \texttt{otherwebsite.com/picture} on it, it
    will replace that with just \texttt{picture} to ensure that the
    browser will use the local versions of that picture
  \item \texttt{-K} This means that \texttt{wget} will make a backup
    of the HTML file when converting links with the \texttt{-k}
    option.
  \item \texttt{-p} is the most important option, as it tells
    \texttt{wget} to download all the requirements as well as the
    url. So if the site links to an outside source (such as
    \texttt{otherwebsite.com/picture}) also gets downloaded if it is
    linked in the requested url.
 \end{itemize}


\end{document}

%[1] https://docs.python.org/3.7/library/multiprocessing.html
%[2] https://www.pcbenchmarks.net/number-of-cpu-cores.html
%[3] https://www.raspberrypi.org/products/raspberry-pi-3-model-b/
%[4] https://www.gnu.org/software/wget/manual/wget.html