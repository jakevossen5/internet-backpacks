\documentclass{article}
\usepackage{graphicx}
\usepackage{hyperref}
\begin{document}

\title{Command and Control Subsystems Report}
\author{Jake Vossen: OREPACKAGERS}

\maketitle

% \begin{abstract}
% The abstract text goes here.
% \end{abstract}

\section{Subsystem Description}
The Command and Control subsystem is the subsystem responsible for
converting the requests that have been collected into downloaded data
to be distributed to users. It starts by receiving a list of
\texttt{request} objects - a structure for contating information about
each request. To prevent confusion, the mono-spaced \texttt{request}
will refer to they Python object itself, wheras plain ``request''
refers to the concept of a user request.

With this list of requests, the first thing it does is use Pythons
\texttt{multiprocessing}[1] library to split work up between the
different threads on the computer. While this software is designed
for low end machines to be more accessable to developing areas, most
computers[2] in recent times will have more than 1 CPU core (including
the Raspberry Pi[3]). This allows for the processor to split up all
the requests, and execute them in paralel, instead of waiting for each
one to finish individually, which can provide a large preformance
boost.

When downloading a \texttt{request}, it determens the type of
request. The types are URL, search, youtube, and ipfs. The steps for
each type of request is outlined below.

\subsection{URLs}

URLs are your basic websites, such as
\url{https://en.wikipedia.org/wiki/Monty_Python_and_the_Holy_Grail},
or
\url{https://www.nytimes.com/2019/03/27/technology/turing-award-ai.html}. This
is for users who already know the content they want. 


\subsection{Subsection Heading Here}
Write your subsection text here.

\section{Conclusion}
Write your conclusion here.

\end{document}

%[1] https://docs.python.org/3.7/library/multiprocessing.html
%[2] https://www.pcbenchmarks.net/number-of-cpu-cores.html
%[3] https://www.raspberrypi.org/products/raspberry-pi-3-model-b/